% This is samplepaper.tex, a sample chapter demonstrating the
% LLNCS macro package for Springer Computer Science proceedings;
% Version 2.20 of 2017/10/04
%
\documentclass[runningheads]{llncs}
%
\usepackage{graphicx}
% Used for displaying a sample figure. If possible, figure files should
% be included in EPS format.
%
% If you use the hyperref package, please uncomment the following line
% to display URLs in blue roman font according to Springer's eBook style:
% \renewcommand\UrlFont{\color{blue}\rmfamily}

\begin{document}
%
\title{Transactions Fraud Detection using Machine Learning and Nature Inspired Algorithms}
%
%\titlerunning{Abbreviated paper title}
% If the paper title is too long for the running head, you can set
% an abbreviated paper title here
%
\author{Peter Mačinec \and Timotej Zaťko}
%
\authorrunning{Peter Mačinec, Timotej Zaťko}
% First names are abbreviated in the running head.
% If there are more than two authors, 'et al.' is used.
%
\institute{Faculty of Informatics and Information Technologies,\\Slovak University of Technology, Bratislava}
%
\maketitle              % typeset the header of the contribution
%
\begin{abstract}
The abstract should briefly summarize the contents of the paper in
15--250 words.

\keywords{Transactions Fraud Detection \and Machine Learning \and Nature Inspired Algorithms \and Data Analysis.}
\end{abstract}
%
%
%
\section{Introduction}

% TODO: co robime, tj, nacrtnut problem atd
% TODO: spomenut, datasety s velkym mnozstvom features, je potrebna feature selection
% TODO: spomenut, ze robi sa to takto takto, ale su aj nejako prirodou inspirovane algoritmy, ktore maju taketo vyhody
% TODO: preco? napr. mzoe chybat domenovat znalost (features) kvoli anonymizacii - nas pripad

\section{Related works}

% TODO: pribuzne prace na feature selection
% TODO: spomenut pouzite algoritmy
% TODO: spomenut domeny, datasety, charakteristky datasetov

- Gray Wolf Optimization Algorithm \cite{Mirjalili_Mirjalili_Lewis_2014}

\section{Problem definition}

% TODO: opisat nas problem
% TODO: spomenut - nevyvazene triedy, velmi vela features, chyba domenovat znalost (features) kvoli anonymizacii
% TODO: opisat dataset - nejaky grafy, pocty pozorovani, features

\section{Method proposal}

% TODO: len kratko, co navrhujeme - feature selection na nejaky model pomocou nejakeho prirodou inspirovaneho algoritmu
% TODO: mozeno spomenut oversampling/undersampling

\section{Preparation}

% TODO: spomenut preprocessing, co sme robili a co z toho vzislo
% TODO: spomenut transformery, one hot, a filter
% TODO: kolko bol finalny pocet features
% TODO: len kratko spomenut ze preco sme sa rozhodli pouzit aky model - decision tree

\section{Experiments}

% TODO: ake experiemnty sme spravili
% TODO: tabulka porovnania algoritmov
% TODO: learning along the way with consequences - napr undersampling/oversampling atd.

\section{Conclusion}

% TODO: zhrnutie vysledkov, mozne vylepsenia a uskalia nasho pristupu

\bibliographystyle{splncs04}
\bibliography{references}

\end{document}
